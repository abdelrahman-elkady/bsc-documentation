\chapter{Conclusion and Future Work}
\label{chap:conclusion}
In this thesis, we explained problems encountered in previous versions of the \ac{se} course. We proposed a new management
system that aims to solve those problems. The system is implemented and tested within the current version of \ac{se} course. We
also displayed analytic results and discussed students feedback. Based on students feedback the system enhanced the announcement delivery for
students. Also, the data collected throughout user registration and team creation helped the course staff with their progress tracking and
course management. The schedule now solves the problem of re-uploading the materials by giving the instructor the flexibility to upload material or
link to external sources. The archiving of questions and staff contact through the discussions section needs more involvement to be tested,
but it is noticed that Slack\texttrademark{} did solve that problem too, thus more research and testing could be conducted
on the connection between the system and Slack\texttrademark{}.

\subsection{Future Recommendations}
\label{sub:future-work}
This project could be enhanced for further usage in the \ac{se} course. Also this system could be extended for another courses,
As shown in the previous chapter, students expressed that they agree to try this system again for other team oriented courses. Some
suggestions for future work is mentioned below.

\begin{itemize}
  \item Project Restructure : This project was initialized with Meteor 1.2.1, during the development of the system, another major release
  \quotes{1.3} is published by Meteor Development Group. The new release recommends a new project structure that would be adopted later in all
  Meteor projects to ensure scalability and modularity.~\cite{meteor_1_3_structure} Restructuring the project and refactoring the codebase would
  support the maintainability and scalability of this system in the future.
  \item Optimization for publication/subscription model : The system needs more optimized publication and subscription model to fetch documents
  from the database onto the client, the current implementation sometimes fetches more data than needed which could cause
  performance problems on larger scale.
  \item Implementing User Interface for instructors to manage grades: Currently the data for the grades is inserted in the database directly with
  server-side scripts. Implementing a user interface to allow the instructor to insert grades would enhance the experience of grade insertion and
  modifications.
  \item Effective handling of user roles : The current implementation assumes that the user role(s) is consistent through all modules of the system.
  While this could be sufficient for now, this could complicate role management if the system started to support different course versions or even different
  courses.
  \item Enhance the notification system : The current notification system does not remove notifications of deleted events. Also, repeated
  events would send repeated notification like up and downvoting the same question multiple times. A better event handling could be implemented
  to avoid duplicate and non valid notifications.
  \item Implement live preview for markdown supported fields : Markdown support in questions, answers, announcements and notifications could support
  live preview. This would enhance the user experience with editing these fields. Also, this would enhance the Slack\texttrademark{} integration experience,
  by having it send only one notification when an update is actually applied to an announcement, not multiple styling updates.
  \item Highlight the updated sections in announcements : Whenever an announcement is edited and updated by the instructor, an option to highlight the changes
  could be implemented, in order to make it easier for students to figure out what changes was applied in the last update.
  \item Support Email Notifications : Users could be provided with options to specify events that they wish to receive Email notifications on.
  \item Implement a reputation system : A reputation system could enhance the experience of the system, motivate students to use the discussions section and help
  each other and enhance the social interaction on the system.
\end{itemize}
