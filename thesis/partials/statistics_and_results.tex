\chapter{Statistics and Results}
\label{chap:statistics}

In this chapter, we show information and data collected from MixPanel\texttrademark{} analytics and from system users feedback, this data shows
overall system satisfaction, user retention, different service rating, effect of updating some services and/or introducing new services on the system.

\subsection{User Retention}
\label{sub:user-retention}
\includefig{0.85}{sems-retention.png}{\ac{sems} User Retention}{fig:sems-retention}

MixPanel\texttrademark{} shows the retention data, the columns shows the returned users in n weeks, it shows that after having a fair stable version
the user retention is increased, it kept stable though with a small difference starting from Mar 28\textsuperscript{th}, this is a bit applicable, given
that users usually comeback when the system have new updates, or when a new announcement or milestone is published, as will be shown in next sections.

\subsection{Page Visits}
\label{sub:page-visits}
\includefig{0.85}{sems-general-graph.png}{\ac{sems} Generic page views}{fig:sems-general-graphs}
Generic page visits graphs can show the interest and the traffic spikes throughout the duration of the course, figure \ref{fig:sems-general-graphs}
shows a sample of 3 frequently visited pages on the system and their visit rate throughout the course, in blue is the home page,
cyan is the team view for members and in purple the team browser, this graphs shows that users are normally having a stable rate of visit, with some
spikes on new feature releases, or when a new announcement is published, in \ref{fig:sems-general-graphs}, the spikes on 29\textsuperscript{th} of March and
12\textsuperscript{th} of April were a new milestone announcement for example, while the spike on the 29\textsuperscript{th} of February was the first
introduction of user registration after a static website stage.

\subsection{Viewing Deployments}
\label{sub:deployments-views}
To measure if allowing teams to share their deployed application with other teams attracted students to see others work or not, we created
a funnel on MixPanel\texttrademark{} that measures the number of students that browses the teams and launches the deployments of other teams during
one month, the results showed that about 60\% of users visiting team browser launches at least one deployment as the first action after visiting the
page as shown in \ref{fig:sems-deployments}. Also we provided a feedback form, where 168 student sent their feedback about the system, we
asked them if it was motivating or demotivating to see other teams deployments, the graph in \ref{fig:sems-deployments-survey} shows that around 70\%
of students found it incentive to see other teams deployments.
\includefig{0.85}{sems-deployments.png}{Users launching deployments after opening team browser}{fig:sems-deployments}
\includefig{0.85}{sems-deployments-survey.png}{Users feedback on sharing deployments on team view}{fig:sems-deployments-survey}

\subsection{Discussions Section}
\label{sub:discussions-stats}
\includefig{0.85}{sems-discussions-graph.png}{Views of discussions section}{fig:sems-discussions-graph}
With the graph shown in \ref{fig:sems-discussions-graph} it is clear that the discussion section attracted many people at the beginning,
however users started to abandon the discussions section as they started to use slack more for questions.

\newParagraph
Although Slack\texttrademark{} integration was notifying about new questions, the bot itself was tested on production by the end
of the course, which can reduce the accuracy of its effect on student involvement in question section. Students feedback on this
section is interesting, as in \ref{fig:sems-discussions-feedback} more than 50\% of students said that the discussions was more
effective than using the email and trying to reach the instructors on campus for example, however more than 65\% of students said
that they never tried to use the discussions for posting questions.
\includefig{0.85}{sems-discussions-feedback.png}{Feedback on discussions section}{fig:sems-discussions-feedback}

\newParagraph
We assume that users needs more motivation to use the discussions section, the Slack\texttrademark{} integration should be tested to see its
effect on interaction with the system, also a reputation and points system could be incentive for students to help each other on the discussions
section.

\subsection{User Profiles}
\label{sub:user-profiles-stats}
\includefig{0.85}{sems-profile-views.png}{User profile views}{fig:sems-profile-views}
The total views of the user public profile shown in \ref{fig:sems-profile-views} shows that the users actually have found it interesting to have
some kind of social interaction, by displaying profiles and linking the user with his actions around the website it seems that users are
interacting with profile links, however, due to the low usage of discussions section discussed in the previous section, the user interaction
could be highly affected, the discussion form could greatly affect the users social interaction, that it is a good idea to re-measure the user
interaction whenever the status of discussions section is changed.

\subsection{Announcements Delivery}
\label{sub:announcements-delivery-stats}
\includefig{0.85}{sems-announcements-stats.png}{Announcements Feedback}{fig:sems-announcements-stats}
Tracking the announcements delivery is important on the system, the users feedback shown in \ref{fig:sems-announcements-stats} shows that the usage
of markdown and code highlighting was liked by many users, also users are satisfied with the delivery of the announcements on the system. That shows
that people are actually interested in having some kind of notifications on announcements, and the look of plain announcements could be boring for
a lot of people and can not provide enough flexibility for content delivery.

\includefig{0.85}{sems-slack-stats.png}{Users Feedback about Slack Integration}{fig:sems-slack-stats}

\newParagraph
In figure \ref{fig:sems-slack-stats} the users are asked about their experience with Slack\texttrademark{} Integration, as expected
many users are happy with Slack\texttrademark{} integration, this supports the previous point that users likes to be notified
without having to check the system everytime for new updates. Interestingly about 25\% are not happy with the integration, this could
be happening because of some problems with the current implementation of the bot that causes a spammy behavior, as the bot posts
about any update of announcements which causes an instructor editing his announcements but not in final stage to send the notification
multiple times, this could be fixed by providing a \quotes{preview edits} functionality for instructors.

\subsection{Overall Feedback}
\label{sub:overall-feedback}
Overall feedback from students shows that the majority of students would like to use this system instead of the \ac{met} portal, that
can show that the development of new systems and/or updating the current course management systems could improve the students satisfaction.
\includefig{0.85}{sems-overall-rating.png}{Overall system feedback}{fig:sems-overall-rating}
