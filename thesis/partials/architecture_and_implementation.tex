\chapter{Architecture and Implementation}
\label{chap:architecture_and_implementation}

In this chapter we have the implementation and the architecture of \ac{sems}, showing the main system modules, design decisions,
problems and impediments and their solutions and the design approaches taken during the implementation of \ac{sems}.

\section{Framework Selection}
\label{sec:framework_selection}
Before starting the implementation of the project, we examined some frameworks, most of them are \ac{js} ( NodeJS ) based;
this is due to the current state of \ac{js} community where one can find a well structured and maintained projects and modules.

\newParagraph
We started to do a quick tests and comparisons between Express.js, Hapi.js, Sails.js and meteor.js; the main goal was to have a stable
and functional system in the shortest period possible, that was required because \ac{sems} was planned to be used in the same year, forcing
us to have a production-ready application within 2 weeks of the project start.

\newParagraph
With that in mind, we chose meteor.js as our main framework, supported with blaze templating engine for the views, and mongoDB as the
database system, allowing us to push forward and focus on system design quickly while maintaining a stable application and framework
to be ready for production releases.

\section{Deployment}
\label{sec:deployment}
We had tested 2 deployment techniques during the implementation of the project, the main goal was to have smooth deployments frequently with
the new features, while maintaining the stability of the system with minimum downtime and quick recovery in case of crashes or bugs.

\newParagraph
First trail was to have a docker container hosting our meteor application on the server, the deployment was easy to be done with a tool
called mupx~\cite{mupx}, However, having a docker container with meteor application was not the best choice for us to recover from downtimes,
that is beacuse we are dealing with local storage on our server as we do not use any external storage server, making it hard to migrate between
different deployments while preserving our local filesystem without problems.

\section{Analytics}
\label{sec:analytics}

\ac{sems} uses MixPanel\texttrademark{} as the main analytic and tracking tool, using analytics.js~\cite{analytics_js} MixPanel\texttrademark{} is
integrated in the system, allowing tracking page views, feature usage and custom events, this improved the evaluation of changes applied on the
system by measuring user interaction and involvement with new features.

\newParagraph
The deployment finally was done using Meteor's build tool, that allows bundling the application and producing an application that is ready
to run in a NodeJS environment, this allowed us to easily build and deploy our application quickly, specially when hotfixes are applied,
and also it helped us to recover from problems that can affect our server filesystem by easily backing up the directories and files frequently.

\section{Student Interaction}
\label{sec:student-interaction}
The source code of this system was hosted on GitHub\texttrademark{} on a public repositories, allowing students to submit issues, help with
suggestions and track the way it was built, also students are provided with an Email to contact the development team about issues related
to the system. During the development of the project some students helped by submitting the issues they faced and providing suggestions.

\section{System Modules}
\label{sec:modules}

In this section, we discover the main modules of \ac{sems}, with implementation notes on specific parts.
\includefig{0.85}{sems-landing-page.png}{\ac{sems} Landing Page}{fig:sems-landing-page}

\subsection{Users}
\label{sub:users}
Users module is the very first module that was constructed on the system, enabling students and instructors to register with their
university emails, and providing basic information that is needed throughout the course, this includes the university ID, the study group
and more contact information for reachability and performance tracking. To ensure that the students are registered with their correct email
address, most system services are locked until the user verifies his email address. A password reset request can be initiated by the user
in case of password loss to reset his password.
\includefig{0.85}{sems-registration.png}{\ac{sems} Registration Form}{fig:sems-registration}

\newParagraph
Each user can access his profile page, edit his basic information and add an avatar to his profile. To have a better connection
between students and instructors, everyone can see the public profile of other users if he is logged in on the system, the user
can determine what to show on his public profile through the profile edit page, also if the user is enrolled in a team, the team
is linked on his public profile automatically.
\includefig{0.85}{sems-public-profile.png}{\ac{sems} user profile}{fig:sems-public-profile}

\newParagraph
\ac{sems} have different roles for users, which is modifiable by administrators, those roles are :

\begin{description}[align=left]
\item [student] The default role upon user creation, a student can be added to teams, and can only create content in discussion section,
student does not have any administrative abilities.
\item [scrum master] A student that have authority over his team, he is allowed to create, add, remove members and modify data of his team.
\item [\ac{jta}] JTAs have access to teams, all announcements and milestones and they can modify and delete questions and answers on discussion section
\item [\ac{ta}] TAs have access to all of the above (Except Team Creation) and they have access to modify schedule and announcements.
\item [Instructor] Instructors have access to all of the above (Except Team Creation) and they can modify and create milestones.
\item [Admin] Admin can modify everything on the system, including removing users and modifying member roles.
\end{description}

\subsection{Course Schedule}
\label{sub:schedule}
\includefig{0.85}{sems-schedule.png}{\ac{sems} Schedule View}{fig:sems-schedule}
The schedule is used for course material posting, it is the only public main service that is accessible without registration in order to make
it easier for students to access it. The Schedule module lists all the course material, including lectures, assignments and code snippets. The schedule
is influenced by the \ac{met} portal course schedule, it displays the materials by the week set in course schedule, also it allows the user
to filter by the material type. The upload problem faced on the \ac{met} portal is solved by allowing the instructor to choose between uploading
the materials or just linking to an outside source, allowing the instructor to have more flexibility to handle the material posted on the system.
Instructors can delete posted materials or edit it on the schedule at any time, changing its details and even changing the type of the material from
uploaded to a direct link and vice-versa.
\includefig{0.85}{sems-schedule-edit.png}{\ac{sems} Schedule Edit}{fig:sems-schedule-edit}

\subsection{Teams}
\label{sub:teams}
\ac{se} is a team oriented course, course layout is structured upon teams, that is why team creation, modification and traversing is an
important module of the system. Team module have multiple views, there is an instructor, scrum master, member and non member view, each
allows access to specific parts of the team based on the role, in the following sections we have a look on the different features and views
in Teams.

\subsubsection{Team Creation}
\label{subs:team-creation}
This view is only accessible by the scrum master if he/she did not create any teams yet, allowing the scrum master to choose the airline company
to work with and choose a name for his team, the scrum master is also asked to provide the GitHub repository of the team that will hold
their codebase.
\includefig{0.65}{sems-team-creation.png}{\ac{sems} Team Creation View}{fig:sems-team-creation}

\subsubsection{Team browse}
\label{subs:team-browse}
When a user navigate through teams tab, all teams are listed in this view, with some information about each team like their deployed application
IP address, their ionic ID \quotes{discussed in \ref{subs:team-settings}} and the number of members in the team, the user can click on a team to
be directed to that team page, there are 3 cases that could happen :
\begin{itemize}
  \item The user is a not a member of the team : The team page will only show the team name, company and the members in the team as in
  figure \ref{fig:sems-team-about}, all other team information is locked and not accessible for that user.
  \item The user is a member of the team : The team page will show the team information, and additionally a side bar is shown
  giving access to milestones, announcements and grades of that team \quotes{discussed in section REFERENCE HERE}.
  \item The user is a member and the scrum master of the team or the user have administrative role : The previous view is shown with an extra
  option to access the team settings panel which is shown in \ref{subs:team-settings}.
\end{itemize}
\includefig{0.65}{sems-team-about.png}{\ac{sems} Team Information}{fig:sems-team-about}
\includefig{0.65}{sems-edit-team.png}{\ac{sems} Team Editing Panel}{fig:sems-edit-team}

\subsubsection{Team Settings}
\label{subs:team-settings}
After creating a team, the team Scrum Master or System Administrators can access the team edit panel, as shown in figure \ref{fig:sems-edit-team}
this panel allows the user to change the team data, add or remove members. The \quotes{Add Members} dropdown displays all users on the system
with a \quotes{student} role who are not enrolled in any team yet. Once a member is added to the team, an Email is sent informing that user
with his team name and that he joined the team successfully, and a badge is added on his public profile linking to the team he had joined,
also, his name is not available to be chosen again in other teams. Users can add the IP of their deployed application, and the ionic ID of their
mobile application to be displayed on their team page, that allows teams to share its public product with each other.

\subsection{Administrator Panel}
\label{sub:admin-panel}
Administrators can access and modify everything on the system, this is done through the administration panel, whenever an admin logs in
a new item is available in the navigation bar for the admin panel shown in figure \ref{fig:sems-admin-panel}
\includefig{0.75}{sems-admin-panel.png}{\ac{sems} Admin Panel}{fig:sems-admin-panel}
\newParagraph
The admin panel have some options that allows the administrator to modify different sections of the system, we are going
to navigate through 2 options which are the user management and announcement management, as the schedule edit is a shortcut to modify
the schedule which was discussed previously, and the other 2 options are related to another bachelor project.

\subsubsection{User Management}
\label{subs:admin-users}
In this page the admin can list all the users on the system, users are searchable, admin can remove users and the role of any user
can be upgraded. The user panel uses a server side search index, in order to load only few users on the client side and request more
users to be loaded, this is discussed in section REFERENCE HERE when we target the security and performance of the system.
\includefig{0.75}{sems-admin-users.png}{\ac{sems} User Management}{fig:sems-admin-users}

\subsubsection{Announcement Management}
\label{subs:announcement-management}
On \ac{sems} instructors can publish announcements and milestones to students, the main difference between them
is that an announcement is a short and informative announcement, and it is expected that announcements are released
more often than milestones. On the other hand, milestones are expected to be a sprint description or a requirement update,
it's expected to be detailed and may contain code snippets, lot of links and lists and are expected to be released infrequently.

\newParagraph
Both Announcements and Milestones allows the usage of markdown syntax, allowing the instructor to insert links, images, emphasize important
parts and mark quoted text. Moreover, code highlighting is supported, enabling addition of code snippets in announcements and milestones. Generally,
code blocks are expected to appear in milestones as mentioned before, however instructors can use inline code blocks to highlight specific parts
in an announcement too.

\newParagraph
Instructor can write markdown directly when he create a new announcement, he/she chooses whether this announcement should be published
as a normal announcement or as a milestone, also, the instructor can specify specific teams to view this announcement/milestone in case
this is specific to some teams only, or the announcement can be denoted as a \quotes{global} announcement which sends it to all
teams.
\includefig{0.75}{sems-admin-announcement.png}{Admin Announcement Panel}{fig:sems-admin-announcement}

\newParagraph
The announcements can be deleted or edited anytime from the admin panel, the editing will affect all the teams who have access to this announcement,
Also, an announcement could be restricted from global to specific teams and vice-versa.

\newParagraph
Whenever an announcement is made, a notification is sent throughout \ac{sems} notification system to all students who are in the target
of the announcement, so, if the announcement is global, a notification would be sent to all members of all teams, and if the announcement
was restricted to some teams, the notification would appear to all members of the targeted teams only, notifications are discussed more in
REFERENCE HERE.

\subsection{Announcements}
\label{sub:announcements}
As mentioned before, the instructors can publish announcements to different teams, the announcements and milestones are accessible through the
team page, announcements are displayed as a stack of non-expandable information, with the date of the announcement. Milestones on the other hand are
displayed as a separate cards holding a short preview of the milestone, and is expandable as shown in \ref{fig:sems-milestone-display} to view
the whole milestone details, this is because milestones are usually longer than announcements, and usually they contain code snippets and sometimes
images.

\includefig{0.75}{sems-announcements-display.png}{Announcement list on team page}{fig:sems-announcements-display}
\includefig{0.75}{sems-milestone-display.png}{Part of a milestone displayed for all teams}{fig:sems-milestone-display}

\subsection{Discussions}
\label{sub:discussions}
The discussions module aim is to reduce the time taken to reach other students or staff when a user seek help, or when he/she needs
clarification about some part in the course, the discussion section is consisting of questions, answers and comments. Mimicking the
design of question and answers services, the discussions section is simple enough, with support to markdown, code highlighting, voting, tags and
best answers marking, it provides some useful tools to ask a question easily. The discussion module briefly consists of the following components :

\subsubsection{Question Browser}
\label{subs:question-browser}
All Questions are listed and paginated in the main discussions page, allowing users to ask new question, search the current questions, up and
downvote questions and check the view and answer count for each question.
\includefig{0.75}{sems-discussions-main.png}{Question Browser}{fig:sems-discussions-main}

\newParagraph
To improve the performance of question navigation and search, the search index is manipulated on the server, allowing the client to subscribe to
the documents in an efficient way by subscribing to the documents on the current page only, and request more whenever the user chooses another page.

\subsubsection{Question Creation}
\label{subs:question-creation}
The user can ask a question by toggling the question creation form displayed in \ref{fig:sems-question-form}, where he can
write his question with markdown support and apply tags, the help icon displays a guide for markdown usage.
\includefig{0.75}{sems-question-form.png}{Question Creation From}{fig:sems-question-form}
