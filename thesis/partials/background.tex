\chapter{Background}
\label{chap:background}

As mentioned before, we are addressing the management of the \ac{se} course, this chapter explains
important definitions and technologies used in the course flow and the system design, in order to understand
the system architecture and implementation which are introduced in the next chapters.

\section{Definitions}
\label{sec:definitions}

\subsection{Scrum}
\label{sub:scrum-process}

Scrum is a methodology and technique for software management and development, introduced in 1995 by Ken Schwaber,
as an extension and enhancement over the normal iterative development model~\cite{Schwaber1997}, the process
provides a lot of flexibility to handle complex system development taking into consideration possible requirement
changes and the effect of changes on software, allowing teams to scale and develop solid software in a more flexible
environment.~\cite{Schwaber1997}

\newParagraph
The Scrum process contains the following definitions, those definitions have more detailed description and contains even
more definitions and points in there, but here we are only focusing on the main definitions of the Scrum process
\begin{itemize}
  \item Scrum Master : The Scrum master is the lead of the development team, as described in the process his job
  is to minimize conflicts between the team and the product owner, hold the retrospective meeting and track
  the progress of the team.~\cite{schwaber_2004}
  \item Product Owner : The project holder, he have the original description of the product and ensure that the
  progress made is adhering the specifications.~\cite{schwaber_2004}
\end{itemize}

\newParagraph
More Scrum components are the development team itself and the documentation and specifications represented into the backlog~\cite{schwaber_2004}.
The course follows the scrum process, requiring each team to be formed by the scrum master, documenting their backlog and hold their meetings, the
technologies used for that is discussed in the next section.

\subsection{Learning Management System}
\label{sub:lms}

A \ac{lms} is a platform that provides administrative services to the content providers and allows delivering, tracking and
documenting the supplied educational content, and track the progress and data of enrolled users of the system~\cite{lms_1}


\section{Technologies and Services}
\label{sec:technologies}

\subsection{Trello\texttrademark{}}
\label{sub:trello}
Trello\texttrademark{} is a simple project management and team collaboration application, launched in 2011\cite{rao_2011}. It attracted a lot
of people because of its simple task orientation as cards and lists, it have been used to manage the scrum process by maintaining the backlog
and distribute the tasks on different lanes~\cite{trello_scrum}\cite{trello_scrum_2}, the backlog and task management is already handled
on Trello\texttrademark{} in this course.

\subsection{Slack\texttrademark{}}
\label{sub:slack}

Slack\texttrademark{} is a team based, cloud hosted messaging application founded in 2013 by Stewart Butterfield, Eric Costello, Cal Henderson
and Serguei Mourachov~\cite{kumparak_2015}. Slack\texttrademark{} is effectively used in teams that follows the agile methodology.

\newParagraph
Moreover, Slack can be used in teaching to allow students to have more freedom and participation in the course discussions, this is a
great environment too for remote education as it allows reaching the instructors easily,~\cite{medium_slack_2015}\cite{slack_education} in the current
version of the course, the students are invited to join a general Slack\texttrademark team for the course to get connected to
other students and instructors.
