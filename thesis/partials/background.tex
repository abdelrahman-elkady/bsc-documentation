\chapter{Background}\label{chap:background}
  Background 
