\chapter{Background}
\label{chap:background}

This chapter explains important definitions and technologies used in the course and system design. Knowledge of those definitions
is required to understand the system architecture and implementation which are introduced in the \ref{chap:architecture_and_implementation}
\textsuperscript{rd} Chapter.

\section{Definitions}
\label{sec:definitions}

\subsection{Scrum}
\label{sub:scrum-process}

Scrum is a methodology and technique for software management and development. Introduced in 1995 by Ken Schwaber
as an extension and enhancement over the normal iterative development model~\cite{Schwaber1997}. The process
provides a lot of flexibility to handle complex system development taking into consideration possible requirement
changes and the effect of changes on software. Scrum allows teams to scale and develop solid software in a more flexible and efficient
environment.~\cite{Schwaber1997}

\newParagraph
The Scrum process contains the following definitions, those definitions have more detailed description and even contain
sub-definitions and points. But here we are only focusing on the main definitions of the Scrum process that are required for project understanding.
\begin{itemize}
  \item Scrum Master : The Scrum master is the lead of the development team. As described in the Scrum model his job
  is to minimize conflicts between the team and the product owner, hold the retrospective meetings and track
  the progress of the team.~\cite{schwaber_2004}
  \item Product Owner : The project holder, he have the original description of the product and ensures that the
  progress made is adhering the specifications.~\cite{schwaber_2004}
\end{itemize}

\newParagraph
More Scrum components are the development team itself, the documentation and specifications represented into the backlog~\cite{schwaber_2004}.
The course follows the scrum process, requiring each team to be formed by the scrum master, documenting their backlog and hold their meetings.

\subsection{Learning Management System}
\label{sub:lms}
A \ac{lms} is a platform that provides administrative services to the content providers and allows delivering, tracking and
documenting the supplied educational content. A \ac{LMS} also tracks the progress and data of enrolled users of the system~\cite{lms_1}


\section{Technologies and Services}
\label{sec:technologies}

\subsection{Trello\texttrademark{}}
\label{sub:trello}
Trello\texttrademark{} is a simple project management and team collaboration application, launched in 2011\cite{rao_2011}. It attracted a lot
of people because of its simple task orientation represented as cards and lists. It had been used to manage the scrum process by maintaining the backlog
and distribute the tasks on different lanes~\cite{trello_scrum}\cite{trello_scrum_2}. The backlog and task management in the scrum
model is managed through Trello\texttrademark{} in this course.

\subsection{Slack\texttrademark{}}
\label{sub:slack}
Slack\texttrademark{} is a team based, cloud hosted messaging application founded in 2013 by Stewart Butterfield, Eric Costello, Cal Henderson
and Serguei Mourachov~\cite{kumparak_2015}. Slack\texttrademark{} is effectively used in teams that follows the agile methodology.~\cite{claps_2015}\cite{davies_2016}

\newParagraph
Moreover, Slack can be used in teaching to allow students to have more freedom and participation in the course discussions. This is a
great environment too for remote education as it allows easier connection with instructors.~\cite{medium_slack_2015}\cite{slack_education} In the current
version of the course, the students are invited to join a general Slack\texttrademark{} team to get connected to other students and instructors.

\subsection{JSON}
\label{sub:json}
\ac{json} is a simple data interchange format. It is designed to be easily read and parsed by both humans and machines.
\ac{json} is used widely now as a data format that most of available languages can decode and encode easily. Given that it is the default
object notation in \ac{js}, it is very common to be used within \ac{js} powered applications.~\cite{json}


\subsection{BSON}
\label{sub:bson}
\ac{bson} is a binary data format that is \ac{json}-like. It uses almost the same grammar as \ac{json}, but it
is encoded into binary, making it more efficient to be stored, manipulated and parsed.~\cite{bson}

\subsection{MongoDB\texttrademark{}}
\label{sub:mongodb}
MongoDB\texttrademark{} is a NoSQL, document based, open-source database system. MongoDB\texttrademark{} became popular and is used in
scalable production applications heavily~\cite{dbs_rank}. MongoDB\texttrademark{} uses \ac{bson} for its document entires, making
it easy to deal with the data in a more modular way that reflects real life modeling.

\subsection{MixPanel\texttrademark{}}
\label{sub:mixpanel}
MixPanel\texttrademark{} is an analytical service that provides a powerful analytic tools for both mobile and web technologies.~\cite{mixpanel}
MixPanel\texttrademark{} provides normal tracking tools to measure events and page visits, and it also provides different analytical
techniques to allow modeling and retrieving information from the data, including tools like funnels and A/B testing.~\cite{mixpanel}

\subsection{SendGrid\texttrademark{}}
\label{sub:sendgrid}
SendGrid\texttrademark{} is an email service provider, providing cloud structured email delivery system. SendGrid\texttrademark{} provides
an API allowing different applications to interact with the system and send emails. It also allows white-listing of email domains and configuration of
\ac{dns} to allow clients to send emails from their own domain.~\cite{sendgrid}

\subsection{NodeJS}
\label{sub:node}
NodeJS is a server-side runtime environment implemented on Google's V8 \ac{js} runtime engine, allowing \ac{js} features to be ran
on the server-side of web applications.~\cite{nodejs}

\subsection{MeteorJS}
\label{sub:meteor}
Meteor is a \ac{js} open-source framework that was founded in 2012~\cite{meteor_launch}.
Meteor focuses on building modern web applications within a narrow time-frame. With its ability to create prototypes quickly, exporting to mobile
and web applications from the same codebase and building and running the application as a normal NodeJS application that can be hosted in a normal
NodeJS environment, it became popular among \ac{js} frameworks.

\subsection{Semantic UI}
\label{sub:semantic_ui}
Semantic UI is a front-end framework. It includes a pre-compiled \ac{css} and \ac{js} files that formulates a set of reusable components
for web and mobile design. The main philosophy behind Semantic UI project is to have an intuitive and human readable components, allowing
the developer to have a more natural experience while building the views of his application.~\cite{semantic_ui}

\section{Previous Work}
\label{sec:previous-work}

\subsection{GUC faculty website - Makeover}
\label{sub:met-makeover-project}
A previous bachelor project that aimed to enhance the view and usability of the \ac{met} portal, also trying to fix the
problems of the system.~\cite{met-makeover}

\subsection{StackExchange}
\label{sub:stackexchange}
StackExchange is a popular \ac{qa} community, allowing people to share knowledge and get answers from professional
people, while archiving answered questions efficiently allowing easy access to answers of old questions, the
platform helps a lot of students and seniors worldwide to reach knowledge easily.
