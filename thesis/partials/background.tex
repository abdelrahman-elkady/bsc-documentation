\chapter{Background}
\label{chap:background}

As mentioned before, we are addressing the management of the \ac{se} course, this chapter explains
important definitions and technologies used in the course flow and the system design, in order to understand
the system architecture and implementation which are introduced in the next chapters.

\section{Scrum}
\label{sec:scrum-process}

Scrum is a methodology and technique for software management and development, introduced in 1995 by Ken Schwaber,
as an extension and enhancement over the normal iterative development model~\cite{Schwaber1997}, the process
provides a lot of flexibility to handle complex system development taking into consideration possible requirement
changes and the effect of changes on software, allowing teams to scale and develop solid software in a more flexible
environment.~\cite{Schwaber1997}

\newParagraph
The Scrum process contains the following definitions, those definitions have more detailed description and contains even
more definitions and points in there, but here we are only focusing on the main definitions of the Scrum process
\begin{itemize}
  \item Scrum Master : The Scrum master is the lead of the development team, as described in the process his job
  is to minimize conflicts between the team and the product owner, hold the retrospective meeting and track
  the progress of the team.~\cite{schwaber_2004}
  \item Product Owner : The project holder, he have the original description of the product and ensure that the
  progress made is adhering the specifications.~\cite{schwaber_2004}
\end{itemize}

\newParagraph
More Scrum components are the development team itself and the documentation represented into the backlog~\cite{schwaber_2004}. The course
follows the scrum process, requiring each team to be formed by the scrum master, documenting their backlog and hold their meetings, the
technologies used for that is discussed in the next section.
