\chapter{Introduction}
\label{chap:intro}

\ac{se} is taught in the \ac{guc} as a core course for \ac{csen}, \ac{bi} students, and as an elective course for \ac{dmet} students, in this thesis
we will be running through an overview of the course, the problems that faced the management process of the course, the main
functionality of the proposed system and the students feedback on the system.

\section{History}
\label{sec:history}
\ac{se} is a team oriented course that requires intensive management and data collection. The course had been through iterations throughout previous years
to reach the current structure. It is mainly focused on software development process, including both technical and project management
technologies and tools. \ac{se} is based on agile development process, which means that teams are created and a scrum master is chosen to manage
and lead the team. The scrum process is discussed more in chapter~\ref{chap:background}.

\section{Motivation}
\label{sec:motivation}

\ac{se} is currently managed through \ac{met} faculty portal, allowing instructor to post materials and
create announcements on the course dashboard. The \ac{met} portal handles course announcements and material listing, However other
managerial processes are managed through different providers. In next sections we are discussing the main items that requires
additional service providers alongside the \ac{met} portal to manage.

\includefig{0.85}{met-course-dashboard.png}{\ac{met} portal course dashboard}{fig:met-course-dashboard}

\subsection{Data Collection}
\label{sub:data-collection}

In order to create teams and track students progress, student data should be collected
at the beginning of the course, including their names, university IDs, github handles, mobile numbers and other info that will
help the instructors to manage the flow of the teams and their work throughout the course. This was usually managed through external services
like Google Forms.

\newParagraph
This approach introduced problems, first, sharing the form with students is usually done via Email,
resulting in reachability problems where some students can not receive the announcement due to Email provider problems.
The other problem is the retrieval of the data, it is not easy for instructors to retrieve
the data collected from the forms later during the course.

\subsection{Team Creation and Management}
\label{sub:team-management}
When the course project is announced, the students are asked to form teams and submit it to the instructors. Team creation
have some generic rules, like having an upper and lower limit for the number of members. Each team should pick
a unique name for representation and team information should be submitted before an announced deadline. After team creation, students
who are not assigned to teams should be randomly assigned to some team. The process of team
creation was also managed with forms or by sending emails to the course staff. This made the team creation a complicated
process for the course staff, specially to manipulate random assignment and ensuring that no member is assigned in multiple teams
by mistake.

\subsection{Course Material}
\label{sub:course-material}
The \ac{met} portal allows instructors to upload material for the course, however it limits the instructor choices as
he must upload the materials. This led some instructors to download their materials from the cloud
to re-upload it or use the announcements section to link to their materials on the cloud.

\subsection{Contacting staff}
\label{sub:contacting-staff}
Students need to ask questions, either course-related or technical ones. Currently students contact the staff using Email or they try to reach them on
campus. The \ac{se} course requires more interaction between students and instructors, which means
it is not enough to use emails. The \ac{met} portal provides a forum section, however it is outdated and is not used anymore.

\newParagraph
Moreover, the instructors have to answer duplicate questions as there is no history tracking of previously answered questions. Some instructors share
a spreadsheet with the students to keep track of old questions and answers, trying to eliminate the duplication. Although this could help reduce the duplication
problem, it yet requires effort to keep the sheet organized and to have students use it correctly.

\section{Main Goals}
\label{sec:main-goals}
Work on this project started after the \ac{se} course had been started already. Given that students have technical sessions before
engaging their practical assignments, we set some goals to achieve starting from first day of the project
\begin{itemize}
  \item Find solutions for the main problems related to course management
  \item Build that system as fast as possible, to not block the course schedule
  \item Release new versions often, allowing more important services of the system to be in production in time
  \item Ensure that sensitive information for both course and individuals is not exposed to everyone
  \item Build a user friendly system for both students and administrators
\end{itemize}

\section{Thesis Organization}
\label{sec:thesis-organization}
\begin{itemize}
  \item Chapter~\ref{chap:intro} - Introduction : In this chapter, we have a quick overview about the course, the problems
  we are trying to solve and the main goals we are maintaining throughout the system design.
  \item Chapter~\ref{chap:background} - Background : This chapter lists the main definitions and introduces the main points that
  are needed to understand the context, technologies and implementation techniques of the system.
  \item Chapter~\ref{chap:architecture_and_implementation} - Architecture and Implementation : In the third chapter we explain the architecture
  and implementation of the system, introducing the technologies used, problems that we encountered and approaches taken during system implementation.
  \item Chapter~\ref{chap:statistics} - Statistics and Results : In this chapter, we have an overall look on the statistics and feedback
  acquired from analytic tools and user feedback. We also discuss the results and the effect of system usage on students.
\end{itemize}
