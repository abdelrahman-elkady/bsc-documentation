\chapter{Introduction}
\label{chap:intro}

\ac{se} is taught in the \ac{guc} as a core course for \ac{csen}, \ac{bi} students, and as an elective course for \ac{dmet} senior students, in the following
sections we will be running through a quick overview of the course, the problems that faced the management process of the course before and the main
functionality of the proposed system.

\section{History}
\label{sec:history}
\ac{se} is a team oriented course that needs a lot of management and data collection, the course had been through many iterations through previous years
to reach the current structure, it is mainly focused on the software development process, including both technical and project management
technologies and tools, \ac{se} is based on the agile development process, which means that teams are created and a scrum master is chosen to manage
and lead the team; The scrum process is discussed later in chapter \ref{chap:background}.

\section{Motivation}
\label{sec:motivation}

\ac{se} was managed through \ac{met} faculty portal, allowing instructor to post materials,
create announcements on the course dashboard, the \ac{met} portal handles course announcements and material listing, However other
mangerial processes are handled through different providers, in the next sections we are discussing the main items that requires
additional service providers than the \ac{met} portal to manage.

\includefig{0.75}{met-course-dashboard.png}{\ac{met} portal course dashboard}{fig:met-course-dashboard}

\subsection{Student Data}
\label{sub:student-data}

In order to start team creation, track the progress of individuals and to reach the students, student data should be collected
at the beginning of the course, including their names, university IDs, github handles, mobile numbers and other info that will
help the instructors to manage the flow of the teams and their work throughout the course, this was usually handled with
forms using an external providers like Google Forms.

\newParagraph
This had some problems, first, distributing the form to studetns, the form was usually sent via Email,
resulting in some problems of reachability when some students can have problem with their Email provider so they can not receive
the announcement, the other problem was the retrival of the data, it was messy and hard for instructors to retrive
the data collected from the forms easily later during the course.

\subsection{Team Creation and Management}
\label{sub:team-management}
When the course project is announced, the students are asked to form teams and submit it to the instructors. Team creation
have some general rules, that is each team have an upper and lower limit for the number of members, each team should pick
a unique name for representation and the teams should be submitted before an announced deadline. After team formation, people
unassigned into teams should be randomly assigned into teams or gathered in a new team by course staff, the process of team
formation was handled again with forms or by sending emails to the course staff, that made the team formation a complicated
process for the course staff, specially with the random assignment and ensuring that no member is assigned in two teams
by mistake.

\subsection{Course Material}
\label{sub:course-material}
The \ac{met} portal allows instructors to upload material for the course, however it limits the instructor choices as
he must upload the materials, resulting into forcing some instructors to download their materials from the cloud
to upload it, or using the announcements section to post links to the materials on the cloud
